% \documentclass[10.5pt,a4j,uplatex]{jsarticle}
\documentclass[10pt,a4j,twocolumn,uplatex]{jsarticle}
% \usepackage{flushend}  % 2段組の最終ページの左右の高さを合わせる
\usepackage[ipaex]{pxchfon}
\usepackage[utf8]{inputenc}
\usepackage{mathtools}
\usepackage{amsmath,amssymb}  % math環境
\usepackage{bm}  % ベクトルを表記
\usepackage[dvipdfmx]{graphicx}
% \usepackage{subfigure}  % 並べた図に異なるキャプションをつける(古いパッケージ)
% \usepackage{verbatim}  % 入力通りの出力
% \usepackage{wrapfig}  % 図の周りに文章を回り込ませる
% \usepackage{ascmac}  % screen環境など
% \usepackage{makeidx}  % ふりがなを付ける
\usepackage{subfiles}  % ファイル分割
% \usepackage{comment}  % コメントアウト
% \usepackage[version=3]{mhchem}  % 化学式
% \usepackage{url}  % url表示
\usepackage[all, error]{onlyamsmath}  % 古いパッケージを警告
\usepackage[dvipdfmx]{hyperref}  % ハイパーリンク
\usepackage{url}  % urlの記載
\usepackage[top=26truemm,bottom=26truemm,left=21truemm,right=21truemm]{geometry}
\graphicspath{{images/}{../images/}}  % 画像読み込みディレクトリ

% 引用コマンド
\newcommand{\tabref}[1]{Table.~\ref{#1}}
\newcommand{\figref}[1]{Fig.~\ref{#1}}
\usepackage{letltxmacro}
\LetLtxMacro{\originaleqref}{\eqref}
\renewcommand{\eqref}{Eq.~\originaleqref}

% 表の表示を英語に
\renewcommand{\figurename}{Fig.}
\renewcommand{\tablename}{Table.}

% abstractの見出し変更
\renewcommand{\abstractname}{}

% 注釈の表示を変更
\renewcommand{\thefootnote}{\arabic{footnote}}

% タイトルの注釈を数字に
\makeatletter%% ここから、おまじない
  \renewcommand{\maketitle}{\par%%% \maketitleの再定義
    \begingroup
%%%      \renewcommand\thefootnote{\@fnsymbol\c@footnote}% オリジナル
        \renewcommand\thefootnote{\arabic{footnote}}%% この行を変更
      \def\@makefnmark{\rlap{\@textsuperscript{\normalfont\@thefnmark}}}%
      \long\def\@makefntext##1{\advance\leftskip 3zw
        \parindent 1zw\noindent
        \llap{\@textsuperscript{\normalfont\@thefnmark}\hskip0.3zw}##1}%
      \if@twocolumn
        \ifnum \col@number=\@ne
          \@maketitle
        \else
          \twocolumn[\@maketitle]%
        \fi
      \else
        \newpage
        \global\@topnum\z@  % Prevents figures from going at top of page.
        \@maketitle
      \fi
      \plainifnotempty
      \@thanks
    \endgroup
    \setcounter{footnote}{0}%
    \global\let\thanks\relax
    \global\let\maketitle\relax
    \global\let\@thanks\@empty
    \global\let\@author\@empty
    \global\let\@date\@empty
    \global\let\@title\@empty
    \global\let\title\relax
    \global\let\author\relax
    \global\let\date\relax
    \global\let\and\relax
  }
  \def\@maketitle{%
    \newpage\null
    \vskip 2em
    \begin{center}%
      \let\footnote\thanks
      {\LARGE \@title \par}%
      \vskip 1.5em
      {\large
        \lineskip .5em
        \begin{tabular}[t]{c}%
          \@author
        \end{tabular}\par}%
      \vskip 1em
      {\large \@date}%
    \end{center}%
    \par\vskip 1.5em
    \ifvoid\@abstractbox\else\centerline{\box\@abstractbox}\vskip1.5em\fi
  }
\makeatother%% ここまで、おまじない
